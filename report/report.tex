% ----------------------------------------------------------------------------------------------------- %
% Manual da Classe UFTeX
% 
% Versão 1.0:   Julho 2024
%
% Criado por:   Tiago da Silva Almeida
%
% https://github.com/UFTeX
% ----------------------------------------------------------------------------------------------------- %

\documentclass[project]{uftreport}
% ---- Esse comando cria o nome uftex estilizado
\newcommand\uftex{UF\TeX}

\usepackage{lipsum}
\usepackage{tikz}
\usepackage[siunitx]{circuitikz}
\usetikzlibrary{arrows}

\usepackage[alf,abnt-emphasize=bf]{../referencias_abnt/abntex2cite}
\renewcommand{\backrefpagesname}{}
\renewcommand{\backref}{}
\renewcommand*{\backrefalt}[4]{}
% ----  Esse comandos são necessário no pré-ambulo para a impressão da lista de lista abreviatuas e de símbolos
%\makelosymbols
%\makeloabbreviations
% ---- Início do documento
\begin{document}
  % ---- Descrição do título do trabalho 
  \title{Laboratório 1: RISCV} 
  % ---- Nome do autor ou autores do trabalho
  \author{Mohamed}{Lee}
  \author{João}{Silva}
  \author{Fulano}{da Silva}
  \author{Ciclano}{Souza}
  
  % ---- Nome do orientador do trabalho. O último campo representa o título do professor
  \advisor{Prof.}{Mohamed}{Lee}{Dr.}
  % ---- Classe é o nome da disciplina ou curso
  \class{Arquitetura de computadores}
  % ---- Departamento representa o curso ao qual o trabalho está sendo apresentado. Descrito por meio de duas iniciais do curso
  \department{LC}
  % ---- Data da apresentação do trabalho
  \date{25}{03}{2024}
  % ---- Palavras-chaves em português do trabalho
  \keyword{\LaTeX}
  \keyword{\uftex}
  \keyword{Trabalho de Conclusão de Curso}
  \keyword{Redação Científica}
  \keyword{Extensão Universitária}
  % ---- Palavras-chaves em inglês do trabalho
  \foreignkeyword{\LaTeX}
  \foreignkeyword{\uftex}
  \foreignkeyword{Bachelor Thesis}
  \foreignkeyword{Scientific Writing}
  \foreignkeyword{University Extension}
  
  % ---- Comando responsável por criar a capa do trabalho, logo em seguida está o comando que insere a folha de rosto conforme a configuração exigida
  \maketitle

% Esse comando Insere a Folha de Rosto
  \frontmatter

  % ----------------------------------------------------------------------------------------------------- %

  \begin{abstract}
  \lipsum[1]
  \end{abstract}

  \begin{foreignabstract}
  \lipsum[9]
  \end{foreignabstract}
  \printlosymbols  
  \printloabbreviations
  % ---- Cria a lista de figuras. OPCIONAL
  \listoffigures
  % ---- Cria a lista de tabelas. OPCIONAL
  \listoftables 
  % ---- Cria o sumário. OBRIGATÓRIO
  \tableofcontents % sumário
% --- Marca o inicio dos elementos textuais. Capítulos.
\mainmatter
% ---- Defino o espaçamento de um e meio centímetros
\onehalfspacing
% ----------------------------------------------------------------------------------------------------- %
% Capítulos do trabalho
% ----------------------------------------------------------------------------------------------------- %
\ChapterStart{first}{First chapter}
\chapter{Introdução}

\lipsum[1] \cite{JW82} \symbl{$\phi$}{Definição do Símbolo 1} \symbl{$\Omega$}{Definição do Símbolo 2}
\lipsum[2] \cite{MenaChalco08} \abbrev{UFT}{Universidade Federal do Tocantins}\abbrev{CUP}{Campus Universitário de Palmas}
\lipsum[3] \cite{alves03:simi}
\section{Foo}
\subsection{Bar}
\citeonline{bobaoglu93:concepts} \lipsum[4]
\subsubsection{Foo}
\lipsum[5] \cite{bronevetsky02}
\lipsum[6] \cite{garcia01:PhD}
\lipsum[7] \cite{schmidt03:MSc}
\lipsum[8] \cite{alvisi99:analysisCIC}

\chapter{Métodos}

\section{Bar}
\lipsum[9] \cite{CORBA:spec}
\begin{eqnarray}
f(x) = \left\{
  \begin{array}{lr}
    x^2 & : x < 0\\
    x^3 & : x \ge 0
  \end{array}
\right.
\end{eqnarray}

\begin{eqnarray}
f(x) & = \int h(x)\, dx
\end{eqnarray}

\begin{eqnarray}
x = a_0 + \frac{1}{a_1 + \frac{1}{a_2 + \frac{1}{a_3 + a_4}}}
\end{eqnarray}


\chapter{Resultados}

\lipsum[15] \cite{waz:09}

\begin{figure}[!htpb]
\centering
\caption{zzzzzzzz.}\label{fig:fig1}
\begin{circuitikz}
	\draw
	% Drawing a npn transistor
	(0,0) node[npn](npn1){} 
	% Making connections from transistor using relative coordinates
	(npn1.E) node[right=7mm, above=5mm]{2N2222} % Labelling the transistor
	(npn1.B) -- ++(-1,0) to [R,l_=10<\kilo\ohm>,*-*] ++(0,-3)  
	(npn1.B) -- ++(-3,0) to [C,l_=100<\nano\farad>] ++(0,-3) node(gnd1){}
	(npn1.E) to [R,l_=10<\kilo\ohm>,*-*] (0,-3)
	(npn1.E) -- ++(2,0) to [C,l=10<\pico\farad>,*-*] (2,-3)
	(npn1.B) -- ++(-1,0) to [R,l_=10<\kilo\ohm>,*-] ++(0,3) node(con1){}
	(npn1.C) to [L,l_=150<\micro\henry>,*-] (0,3) 
	(npn1.C) -- ++(2,0) to [C,l=10<\pico\farad>,*-*] ++(0,-1.5)
	% Drawing shorts and ground connection
	(-1,3)to[short,*-o] (-1,4) node[right]{$V_{DD}=6 VDC$} % Power supply
	% Output sinusoidal waveform at approximately 50 MHz
	(npn1.C) -- ++(4,0) to [short,-o]
	  ++(0,0) node[right]{$V_o (\approx \SI{50}{\MHz})$}
	(0,-3) node[ground]{}% Define this node as ground
	(gnd1) ++(0,0) to[short,-o] ++(7.85,0)
%       (con1)to[short] ++(1.85,0)
	;
\end{circuitikz}
\end{figure}


% ----------------------------------------------------------------------------------------------------- %
\bibliography{report}

\appendix
\onehalfspacing

\chapter{Sequências}
\label{ape:sequencias}

\lipsum[7]


\end{document}
